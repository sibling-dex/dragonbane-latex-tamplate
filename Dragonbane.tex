\documentclass[
    %blackwhite
]{dragonbane-supplement}
%% ================ TITLE ================ %%
\title{Dragonbane Supplement}
\subtitle{A LaTeX-Template for Dragonbane}
\author{Sibling Dex}
\version{2 (2025-05-15)}
\credits{
    Many thanks to Fahien for feedback and code contribution.
    
    \href{https://www.deviantart.com/esther-sanz/art/Old-Scroll-Texture-II-114214631}{"Old Scroll Texture"} by Esther Sanz(\href{http://creativecommons.org/licenses/by/3.0/}{CC BY 3.0})

    "The Jabberwocky" by John Tenniel (Public Domain).
    
    This game supplement was created under Fria Ligan AB’s \href{https://freeleaguepublishing.com/wp-content/uploads/2023/11/Dragonbane-License-Agreement.pdf}{\emph{Dragonbane Third Party Supplement License}} to be used with the core rules of \textit{Dragonbane}.
    
    This game supplement is neither affiliated with, sponsored, or endorsed by Fria Ligan AB.
}
%% ================ DOCUMENT ================ %%
\begin{document}
\maketitle

\partcolor{DragonRed} % Can be used to change the accent color
\partimage{img/Jabberwocky.jpg}
\part{Instructions}

\partcolor{DemonGreen} % The default accent color
\chapter{DocumentClass Features}

\begin{segment}

\begin{quote}
    "Quotes are useful to provide a short and flavourful introduction to a topic at the top of a new section or chapter."

    --The Author (always give credit)
\end{quote}

\subsection{Parts}
The \texttt{\textbackslash part} command generates a full-page heading with space for an image, as seen on the page before this one. Use the \texttt{\textbackslash partimage} command to set the image used on the page. Parts are also a good level to use the \texttt{\textbackslash partcolor} command to set the accent colour featured in many of the class's elements.

\subsection{Chapters}
The \texttt{\textbackslash chapter} command produces a large, accent-coloured, fancy title, such as the one on this page. Apart from its decoration, the chapter heading looks like the other section titles, set by the \texttt{\textbackslash segment} environment.
\end{segment}


\begin{segment}[Segments]
The \texttt{\textbackslash segment} environment generates a two-column layout. This is an easier way than the \texttt{\textbackslash twocolumn} option to create a two-column document with many column-spanning elements, such as tables. Remember to end your segments. Otherwise, you will get error messages.

\paragraph{With Heading:}
Segments take an optional argument that generates a column-spanning \texttt{\textbackslash section} heading in green, as above. This is the main way to generate section headings, alternatively, you can also use the \texttt{\textbackslash section} command. In that case, be careful to place it \emph{outside} any \texttt{\textbackslash segment} environment, or you will get an oddly placed, one-column title.

\paragraph{Without Heading:}
The optional argument can also be omitted to generate no heading, as after the table below.

\subsection{Subsections}
The \texttt{subsection} command generates the standard bold-faced, left-aligned heading you see above this paragraph. Use them to divide your longer texts into smaller chunks to provide a nice orientation for the reader.

\subsection{Paragraphs}
The smallest defined division in this document class is the \texttt{paragraph}. This command generated an inline bold heading that is best used for an unnumbered list of entries that are too long to put in an actual list environment.

\paragraph{This is a Paragraph:}
A nice way to use paragraphs is to add a colon at the end of the heading, as seen here.


\subsection{Lists}
You can use a normal \texttt{itemize} environment and \texttt{\textbackslash item} commands inside a list to produce a simple list entry.

\begin{itemize}
    \bolditem{Bold Item:} To add a bold keyword to the beginning of an item, use the \texttt{\textbackslash bolditem\{...\}} command.
    \coloritem{Color Item:} To make the bold keyword green, use the \texttt{\textbackslash coloritem\{...\}} command. This is often used in lists that list the important aspects of a location.
    \secretitem{Secret Item:} You can make the keyword red and italic by using the \texttt{\textbackslash secretitem\{...\}} command. This is often used to list a secret or hidden feature of a location.
    \coloritem[blue]{Color Item:}  You can make the keyword a custom color by using the \texttt{\textbackslash coloritem[color]\{...\}} command and specify the color in the square bracket.
\end{itemize}
\end{segment}


%% ================================================================ %%

\begin{segment}[Package Options]
\subsection{Generic Options}
You can use all class options available to the standard \texttt{report} document class included in LaTeX.

\paragraph{Papersize:} You can use the class with paper sizes other than A4, but be sure to keep the \texttt{textwidth} at 16cm, otherwise the headers will get uncentered.

\subsection{Black and White}
You can use the \texttt{blackwhite} option in the \texttt{documentclass} command to generate the document in black and white colours. When doing so, you have to manually set monochrome images, the option cannot change included images.
\end{segment}

%% ================================================================ %%

\begin{segment}[Boxes]

There are three types of box provided by this class: the Demonbox, Dragonbox and Emptybox. These are special environments that can be used to highlight special rules or important information in a compact way. These boxes are not floats but are placed as part of the text. Therefore, they can be placed both inside a \texttt{segment} environment, to produce a one-column wide box, or outside, to create a two-column spanning box.

\begin{dragonbox}{Dragonbox}
\begin{itemize}
    \item \textbf{These Rules Are:} Obligatory
\end{itemize}
    This is a \texttt{dragonbox}, it can be used to highlight important information in a compact and noticeable way.
    
    It can be used, for example, to typeset a Heroic Ability. In that case, you can use an \texttt{itemize} list, as above, to note the Willpower cost for the ability.
\end{dragonbox}

\begin{emptybox}{Emptybox}
    This is an \texttt{emptybox}, it features the same heading as a \texttt{demonbox} but not the coloured background. It is used in the Tablebox environment but can also be used by itself.
\end{emptybox}

\begin{demonbox}{Demonbox}
\begin{itemize}
    \item \textbf{These Rules Are:} Optional
\end{itemize}
    This is a \texttt{demonbox}, they are used to add information about optional rules.

    The \texttt{demonbox} is fully coloured in the current accent colour and can be a drain on printer toner or ink. It is generally a good approach to use these boxes sparingly. A restrained use of boxes in general also prevents the layout from looking cluttered.
\end{demonbox}
\end{segment}


\begin{dragonbox}{Wide Boxes}
\begin{segment}
Both \texttt{dragonbox} and \texttt{demonbox} can be used outside a \texttt{segment} environment to make it span the whole page width. When using text inside a wide box, it is good practice to use a \texttt{segment} environment inside the box to get a two column layout in the box and prevent overly long lines. 

\subsection{Subsection}
Lower level headings such as \texttt{subsection} and \texttt{paragraph} can be used inside boxes.

\paragraph{Paragraph:}
Using these headings can help make the text inside a box more ordered and provide a better overview.
\end{segment}
\end{dragonbox}


\begin{tablebox}{Tablebox}
\begin{tabulary}{\linewidth}{ c l c L }
    \textbf{DICE} & \textbf{LABEL} & \textbf{ALIGNMENT} & \textbf{DESCRIPTION} \\
    \hline
    1 & Dice & Center (c) & If you want a table to be rollable, use the first column as the die or dice column. Give it a header denoting the die/dice used, and number the rows. \\
    2 & Label & Left (l) & The first or second column of a table should be the label of the entry. This gives a short and meaningful name to the entry in the row. \\
    3 & Score & Center (c) & You can add several narrower columns for short, standardized scores, such as price, availability, durability, etc. \\
    4 & Description & Left (L) & The typically last column in a table is a longer description of the entry. Use a breaking alignment for this, so the description can be more than one line. \\
\end{tabulary}
\end{tablebox}

\end{document}
